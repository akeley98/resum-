\documentclass[11pt]{article}
\usepackage[letterpaper, portrait, margin=20mm]{geometry}

\usepackage{enumitem}
\usepackage{amsmath}
\usepackage{placeins}
\usepackage{graphicx}
\usepackage{caption}
\usepackage{xcolor}
\usepackage{colortbl}
\usepackage[parfill]{parskip}
\usepackage[hidelinks]{hyperref}
\usepackage{fontspec}
\linespread{1.0}
\setlength{\parskip}{1mm}

\setsansfont{FreeSans}
\setmonofont{Ubuntu Mono}

\hyphenation{WebGL}

\definecolor{webColor}{RGB}{0, 72, 225}
\newcommand{\web}[1]{{\color{webColor} \small \url{#1}}}
\newcommand{\webText}[2]{{\color{webColor} \href{#1}{#2}}}
\newcommand{\email}[2]{{\small \color{webColor} \textsf{\href{mailto:#1@#2}{#1[at]#2}}}}
\definecolor{titleColor}{RGB}{0, 0, 0}
\ifdefined\RESUME
\newcommand{\myTitle}[1]{{\large \color{titleColor} \hspace{-12mm} \textbf{\textsf{#1}}}}
\else
\newcommand{\myTitle}[1]{{ \vspace{2mm} \large \color{titleColor} \hspace{-12mm} \textbf{\textsf{#1}} \vspace{2mm}}}
\fi
\definecolor{subColor}{RGB}{170, 0, 149}
\newcommand{\mySub}[1]{{\color{subColor}\hspace{-6mm} \textsf{#1}}}
\definecolor{keyColor}{RGB}{170, 149, 0}
\newcommand{\myKey}[1]{{\color{keyColor}\textbf{#1}}}
\definecolor{oddColor}{RGB}{242, 236, 248}
\definecolor{evenColor}{RGB}{248, 248, 248}
\definecolor{oddTextColor}{RGB}{92, 0, 92}

\newcommand{\GPA}{3.762}
\newcommand{\GpaDate}{August 2020}
\definecolor{lightttColor}{RGB}{69, 69, 80}
\newcommand{\lighttt}[1]{{\color{lightttColor}\texttt{#1}}}
\newcommand{\oddQuarter}[1]{{\color{oddTextColor}\hspace{-20mm} \textsf{#1}}}
\newcommand{\evenQuarter}[1]{{\hspace{-20mm} \textsf{#1}}}

\ifdefined\RESUME
\newcommand{\resumeMiniItemURL}[2]{\hfill -- \hfill\web{#1}\\#2}
\newcommand{\resumeMiniItem}[1]{\\#1}
\else
\newcommand{\resumeMiniItemURL}[2]{}
\newcommand{\resumeMiniItem}[1]{}
\fi

\ifdefined\CV
\newcommand{\cvFootnote}[1]{\footnote{#1}}
\else
\newcommand{\cvFootnote}[1]{}
\fi


\begin{document}
\raggedright
\reversemarginpar
\begin{center}
\textsf{\textbf{Statement of Purpose Outline/Notes}}
\end{center}

Based on \web{https://grad.berkeley.edu/admissions/steps-to-apply/requirements/statement-purpose/}

\myTitle{Introduction}

\begin{enumerate}
\item Background: several years experience in parallel computing, and
  designing and implementing real-world systems.
\item I'm interested in new applications for parallel and accelerated
  computing, and potentially researching disciplined ``Halide-like''
  abstractions to make this possible.
\end{enumerate}

\myTitle{Research Experience}

\mySub{NVIDIA Research}

\begin{enumerate}
\item Research on novel GPU algorithms for 2D computational geometry,
  based on signed distance fields.
\item Targetting applications in PCB design software, manufacturing, lithography.
\item Working with Mark Kilgard (NVIDIA Research), as well as outside
  customers, but the vast majority of algorithms were invented and
  implemented by me.
\item I wish I can give more details on this but this is not really
  public right now, besides my very high-level talk below:
\end{enumerate}

``How to Accelerate 2D Shape Processing for Manufacturing and Planning''
- \webText{https://www.nvidia.com/en-us/on-demand/session/gtcspring23-s51140/}{GTC 2023}
(s51140)
\filbreak

\mySub{Stanford FPGA Research}

\begin{enumerate}
\item Undergraduate co-author of Aetherling with
  \webText{https://david-durst.github.io/}{David Durst} (lead author),
  \webText{https://graphics.stanford.edu/~kayvonf/}{Dr. Kayvon Fatahalian},
  and \webText{https://graphics.stanford.edu/~hanrahan/}{Dr. Pat Hanrahan}.
\item Research into compiling from a high-level Haskell description of
  an image-processing pipeline into an FPGA design; by using a type
  system that encodes timing information, the design is automatically
  parallelized to utilize the amount of resources available.
\item Prototyped solutions (in Haskell) to functional simulation, line
  buffer parallelization, the type system, and static ``retiming'' (to
  avoid ready-valid interfaces).
\end{enumerate}

\quad\web{https://aetherling.org} (``Type-Directed Scheduling of Streaming Accelerators'' - PLDI 2020)
\filbreak

\mySub{MediocrePy -- Astrophysics Number-crunching Project}

\begin{enumerate}
\item Worked with
  \webText{http://sancerre.as.arizona.edu/~caiz/Home/Welcome.html}{Dr. Zheng Cai},
  (UC Santa Cruz Astrophysics) to understand their performance
  problems and computation needs.
\item Implemented an extensively-optimized multithreaded AVX solution
  (sole software author).
\item Problem: Combine a short stack of N-many 2D images (N is several
  hundred) into a single image, where the output $(x,y)$ pixel is
  found by applying a pixelwise ``combine'' function to
  the corresponding stack of input pixels $\{(x,y,i) \mid i = 0, 1,..., N-1 \}$
\item Non-trivial combine functions: median, custom ``scaled mean''
  algorithm, all with iterative sigma-clipping outlier rejection.
\end{enumerate}

\quad\web{https://github.com/akeley98/MediocrePy}
\filbreak

\myTitle{Work Activities}

I've been working at NVIDIA for three years as an IC3-level developer
technology engineer (IC1-level when hired). I work internally with
NVIDIA research and outside with independent software vendors. Most of
my work is in CUDA, but I am also experienced with modern
\lighttt{C++}, OpenGL, and have published educational samples with
Vulkan as part of my professional work.

\filbreak

\myTitle{Interests}

\begin{enumerate}
\item Short-term: new applications for parallel programming in
  research, such as new applications in scientific computing.
\item Longer-term/aspirational: parallel programming abstractions such
  as Halide enable incredible productivity in optimizing parallelized
  software in a highly-\textit{disciplined} (i.e. provably-correct)
  manner; however, they are generally limited to image-processing or
  image-processing-like software (e.g. machine
  learning). \textit{(Fact check?)} Can such a disciplined and
  optimized abstraction be adapted to (or inspire) a more
  general-purpose solution? (or at least new DSLs developed for other
  fields?)
\item Example: I didn't see a way to fit this model to the
  astrophysics software (MediocrePy) I developed (in particular, the
  sigma-clipping outlier-rejection logic required highly-nontrivial
  custom logic in the ``inner loop'' of the parallelization). This
  software solution exists only because of a chance social encounter
  between me and Dr. Zheng Cai. Imagine the productivity that would be
  enabled if abstactions existed that allowed scientists like him,
  with non-expert programming skills, to develop similarly
  high-performance software without the involvement of someone like
  me!
\item I hope to continue to develop carefully-architected, useful, and
  maintainable systems as part of my future research work.
\end{enumerate}

{
\vfill
\begin{center}
\color{gray}
Typeset in \LaTeX\

Fonts: Computer Modern, \textsf{FreeSans} (GNU FreeFont),
\texttt{Ubuntu Mono} (Dalton Maag \& Canonical Ltd.)
\end{center}
}

\end{document}

% Local Variables:
% mode: tex
% End:
