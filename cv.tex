\documentclass[11pt]{article}
\usepackage[letterpaper, portrait, margin=20mm]{geometry}

\usepackage{enumitem}
\usepackage{amsmath}
\usepackage{placeins}
\usepackage{graphicx}
\usepackage{caption}
\usepackage{xcolor}
\usepackage[parfill]{parskip}
\usepackage[hidelinks]{hyperref}
\usepackage{fontspec}
\linespread{1.0}
\setlength{\parskip}{1mm}

\setsansfont{Ubuntu}
\setmonofont{Ubuntu Mono}

\hyphenation{WebGL}

\definecolor{webColor}{RGB}{0, 110, 180}
\newcommand{\web}[1]{{ \color{webColor} \small \url{#1}}}
\definecolor{titleColor}{RGB}{170, 30, 120}
\newcommand{\myTitle}[1]{\vspace{2mm}{ \LARGE \color{titleColor} \textsf{#1}\vspace{2mm}}}
\definecolor{subColor}{RGB}{170, 96, 120}
\newcommand{\mySub}[1]{{\color{subColor}$\times$\hspace{1mm}\textsf{#1}}}
\definecolor{keyColor}{RGB}{100, 80, 30}
\newcommand{\myKey}[1]{{\color{keyColor}\textbf{#1}}}

\newcommand{\GPA}{3.745}
\newcommand{\GpaDate}{July 2020}

\begin{document}

\begin{center}
\myTitle{David Zhao Akeley}

\vspace{-2.5mm}
{ \color{subColor} \textsf{UCLA Engineering undergraduate} }
\end{center}


\myKey{Majors} Computer Science, Mathematics
\hfill \myKey{GPA} \GPA
\hfill \myKey{Expected Graduation} August 2020

\myKey{Select Engineering Courses} Parallel \& Distributed Computing,
Advanced Computer Architecture

\myKey{Select Mathematics Courses} Complex Analysis Honors, Algebra Honors
\vspace{3mm}

\myTitle{Work Experience}

\mySub{Sholari LLC -- July - September 2019 -- Contractor}
\begin{enumerate}
\item Worked on a Unity 3D game that simulates tumor growth and
  provides visualizations of tumor response to various treatment
  options.
\item Implemented line graphs, waterfall (bar) plots, and the user
  interface for the timeline (graph x-axis).
\item Wrote a multithreaded \texttt{C++11} plugin for visualizing
  tumors \& immune system responses as particle clouds, and integrated
  it with the single-threaded \texttt{C\#} Unity Engine.
\end{enumerate}

\mySub{Stanford Aetherling Project -- June - September 2018 --
  Research Assistant}
\begin{enumerate}
\item Aetherling supports automatic parallelization and static
  scheduling\footnote{i.e. not using ready-valid hardware interfaces.}
  of hardware image processing pipelines designed using a
  Haskell-embedded domain-specific language.
\item Contributed to the functional simulator of an early prototype of
  Aetherling and worked to remove impediments to parallelizing the
  prototype's line buffers.\footnote{A line buffer device reads in an
    image as a stream of pixel values and outputs rectangular portions
    (``windows'') of the image.}
\item Collaborated with David Durst (lead author), Dr. Kayvon
  Fatahalian, and Dr. Pat Hanrahan.
\end{enumerate}
\web{https://aetherling.org}

\web{https://github.com/David-Durst/embeddedHaskellAetherling}

\web{https://github.com/David-Durst/aetherlingHaskellIR} (early prototype)

\mySub{MediocrePy -- March - June 2017 -- Independent Project}
\begin{enumerate}
\item Created an optimized library for reducing stacks of telescope
  images to a single image using pixel means or medians and optional
  outlier rejection (sigma clipping) for noise reduction.
\item Multithreaded C core with AVX vectorization; C and Python
  (numpy) interface. Decreased runtime (compared to the Python
  implementation replaced) from hours to milliseconds.
\item Collaborated with Dr. Zheng Cai, UC Santa Cruz Astrophysics.
\end{enumerate}
\web{https://github.com/akeley98/MediocrePy}

\mySub{Tsinghua Astrophysics -- July - August 2016 -- Summer Intern}
\begin{enumerate}
\item Designed a library for fitting and plotting standard
  microlensing event light curves given a set of brightness
  data for a star.
\item Used Python, \texttt{C++}, SciPy, Matplotlib.
\item Collaborated with Dr. Shude Mao.
\end{enumerate}

\mySub{Jide Technology Co. -- June - July 2015 -- Summer Intern}
\begin{enumerate}
\item Product testing for RemixOS, an Android derivative with a
  desktop-like interface.

\item Wrote international marketing \& documentation in English.

\item Collaborated with Jason Zheng and Jeff Zhao (International
  Marketing Manager).
\end{enumerate}

\newpage
\myTitle{Other Projects}

\mySub{WebGL Jelly Cube Project}

Simple mass-spring system simulation written with Javascript,
WebAssembly, and WebGL 2.0 (for refractive and reflective effects).
Earned third place in the UCLA computer graphics class contest, Fall
2017.\footnote{
  \web{https://www.facebook.com/vasilescu.alex/posts/10155206917936588}}

\web{https://github.com/akeley98/JellyMcJelloFace}
\hfill\web{https://youtu.be/YwvMSeB6NzU}

\mySub{DementedIGPU -- Linux Nvidia Setup Script}

(Unfortunately, this project no longer works due to Bumblebee's
end-of-support).

Laptops with Nvidia graphics cards often work unreliably with the
GNU/Linux operating system, especially when attempting to switch
between high-performance discrete graphics and low-power integrated
graphics. I wrote a Python 3 script that automatically installs and
configures software needed to provide a (relatively) reliable option
at boot time between high- and low-power graphics.\footnote{This
  automation depends on the user using a system with \texttt{apt},
  \texttt{systemd}, and the \texttt{GRUB} bootloader. Tested with
  Ubuntu 18.04.} I documented the script liberally in order to make it
as beginner-friendly as a command line application can be.

\web{https://github.com/akeley98/DementedIGPU}

\mySub{Myricube -- Experimental OpenGL Voxel Renderer (WIP)}

This experiment in hybrid raycast/mesh voxel renderering was inspired
by my discontent with Minecraft's low render distance, even on high
settings.

Conventional voxel renderers such as Minecraft's work by converting
voxel models into a mesh of triangles to draw, possibly with some
optimizations like hidden face removal or merging coplanar
triangles. This uses the GPU for its designed purpose, but as render
distance increases, the GPU is bottlenecked by the large number of
small triangles generated. Myricube uses conventional mesh rendering
for nearby voxels but draws more distant voxels with an alternative
raycasting renderer. This renderer subdivides the voxel model into
cubic chunks, computes a minimal AABB\footnote{Axis aligned bounding
  box -- requires only 12 triangles to draw.} for each chunk's visible
voxels, and draws the AABBs themselves with a raycasting fragment
shader that only checks for intersections within the drawn AABB.
Dividing the model into AABBs reduces my renderer's memory bandwidth
requirements and time wasted raycasting through empty space, compared
to a conventional raycasting renderer.

\web{https://github.com/akeley98/myricube}

\mySub{Proposed gem5 Hardware Simulator Partial Bypassing Patch}

This patch is a refactoring of the \texttt{C++} code I wrote as part
of my UCLA ``Advanced Computer Architecture'' course project.

In an out-of-order CPU, hardware bypassing allows instructions stalled
waiting for a register to start execution as soon as the functional
unit (ALU, memory port, etc.) writing to said register completes,
without waiting for the actual commitment to register file. This is
crucial for performance, so mainline \texttt{gem5} simulates complete
bypassing between all functional units in the simulated CPU. However,
this is not practical to realize in modern superscalar hardware, as
bypassing costs grow quadratically with the number of functional
units. My patch splits the CPU's functional units into separate pools,
with bypassing simulated only between pools. The
commit-to-register-file delay is imposed for values communicated
between pools.

Although the patch received positive code reviews, it doesn't look
like there's enough interest in the feature to actually merge it into
mainline \texttt{gem5}.

\web{https://gem5-review.googlesource.com/c/public/gem5/+/27767}

\web{https://github.com/akeley98/FU-pools/blob/b4d291429edb6aa4988888656b6867ff99591b90/fu-pools.pdf}

\newpage
\myTitle{UCLA Education -- September 2017 - August 2020 (Expected)}

\myKey{First Major} Computer Science \hfill\myKey{Second Major}
Mathematics \hfill\myKey{GPA} \GPA\ (\GpaDate)


% Note classes organized by name, not by time.
% \textbf{Engineering Courses}

\begin{tabular}{l l l}
\hline
 & Title (\textit{In Progress}) & Content Notes \\
\hline
EE M16 & Digital Systems & Verilog Lab \\
EE M116C & Computer Systems Architecture & \\
CS M152A & Digital Design Lab & Verilog Team Project \\
CS 251A & Advanced Computer Architecture & \texttt{gem5} Hardware Sim Project, Graduate Course \\
\hline
CS 35L & Software Construction Lab & POSIX basics (e.g. pthreads, bash) \\
CS 111 & Operating Systems Principles & Focus on POSIX \\
CS 118 & Computer Network Fundamentals & \\
CS 130 & Software Engineering & Java Team Project \\
CS 131 & Programming Languages & \\
CS 133 & Parallel \& Distributed Computing & OpenMP, OpenCL, MPI, GPGPU, FPGA \\
CS M146 & Machine Learning & \\
CS 161 & Fundamentals of Artificial Intelligence & \\
CS 174A & Intro to Computer Graphics & See WebGL Jelly Cube Project \\
CS 180 & Algorithms \& Complexity & \\
CS 181 & Formal Languages \& Automata & Regex, CFG, Turing Machines, Decidability \\
\hline
Engr 185EW & Art of Engineering Endeavors & Writing Intensive Team Project \\
\hline
%% \end{tabular}
%% %\footnotetext{Full title: Linear and Nonlinear Systems of Differential Equations}
%% %\footnotetext{Full title: Introduction to Formal
%% %Languages and Automata Theory}

%% \newpage
%% \textbf{Mathematics Courses}

%% \begin{tabular}{l l l}
%% \hline
%%  & Title (\textit{In Progress}) & Content Notes \\
%% \hline
Math 110A & Algebra & Ring Theory \\
Math 110AH & Algebra Honors & Group Theory \\
Math 110BH & Algebra Honors & Ring Theory, Module Theory \\
Math 110C & Algebra & Field Theory, Galois Theory \\
Math 111 & Theory of Numbers & Overview of p-adic Numbers \\
Math 115A & Linear Algebra & \\
Math 115B & Linear Algebra & \\
Math 120A & Differential Geometry & \\
Math 131AH & Analysis Honors & Metric Spaces \\
Math 131BH & Analysis Honors & Derivation, Riemann Integration \\
Math 132H & Complex Analysis Honors & \\
Math 134 & Systems of Differential Equations & \\
\textit{Math 142} & \textit{Mathematical Modeling} & \\
\textit{Math 167} & \textit{Game Theory} & \\
Math 170A & Probability Theory & \\
\hline
\end{tabular}

\textit{Note: There is no honors equivalent to the Field Theory Course.}

\myTitle{West Valley College Education -- 2015-2017}

\myKey{GPA} 4.0 (upon transferring to UCLA)

\myKey{Select Courses}

\begin{tabular}{l l l}
\hline
 & Title & Content Notes \\
\hline
Math 4B & Differential Equations & \\
Math 19 & Discrete Mathematics & \\
Psych 2 & Experimental Psychophysiology & Experiment Design \& Paper \\
Phys 4D & Modern Physics & Relativity \\
\hline
\end{tabular}

\vspace{10mm}
{
\begin{center}
\textsf{Ubuntu}, \texttt{Ubuntu Mono} font by Dalton Maag and Canonical Ltd.
\end{center}
}

\end{document}

% Local Variables:
% mode: tex
% End:
